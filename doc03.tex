% PREAMBLE
\documentclass{article}

\usepackage{amsmath}


\title{The Pythagoras formula explained}
\author{Pythagoras}
\date{12.03.2025}

% DOCUMENT 
\begin{document}
\pagenumbering{gobble}
\maketitle
\newpage
\pagenumbering{arabic}

\section{Introduction}
Why the Pythagoras formula? Its always a nice to have and nice to brag about knowing the formula. This is why I (Pythagoras invented it). It may be used but before all things it is a very biutiful formula. 

\section{Lets get into the math}
\paragraph{Variables} 
The important variables for this part are \(a\), \(b\) and \(c\).
\paragraph{The idea}
The idea is to make a relation between the surface of three squares and sadly it does not work for volumens or the like.
\paragraph{ The formula}
\subparagraph{First without number:}
\begin{equation*}
  a^2 + b^2 = c^2 
\end{equation*}
\subparagraph{Now with numbering!}
\begin{equation}
  a^2 + b^2 = c^2 
\end{equation}
\end{document}

